\documentclass[a4paper, 12pt]{article}
\usepackage{xeCJK}
    \setCJKmainfont[AutoFakeBold=1,AutoFakeSlant=.4]{標楷體}
    \XeTeXlinebreaklocale "zh"
    \XeTeXlinebreakskip = 0pt plus 1pt
\usepackage{fontspec}
    \setmainfont{Times New Roman}
\usepackage{setspace}
    \onehalfspace
    \setlength{\parskip}{1ex plus 0.5ex minus 0.2ex}
\usepackage{mathtools}
\usepackage{graphicx}
\usepackage{enumitem}
\usepackage{xcolor}
\parindent=0pt
\def\Large{\fontsize{16}{24}\selectfont}
\def\large{\fontsize{14}{20}\selectfont}
\makeatletter
\renewcommand\section{\@startsection {section}{1}{\z@}%
                                   {-3.5ex \@plus -1ex \@minus -.2ex}%
                                   {2.3ex \@plus.2ex}%
                                   {\centering\normalfont\Large\bfseries}}
\renewcommand\subsection{\@startsection {subsection}{1}{\z@}%
                                   {-3.5ex \@plus -1ex \@minus -.2ex}%
                                   {2.3ex \@plus.2ex}%
                                   {\centering\normalfont\large\bfseries}}
\makeatother
\begin{document}
\section*{分析目的}
空氣汙染,一直是人們關心的議題,它對(我們的健康狀況有一定的影響),在(某些地區)尤其嚴重。空氣汙染不單指某一個特定汙染物,而是許多汙染物的總稱,這些物質可以是固體,像是懸浮微粒PM2.5、PM10,或是氣體,例如:一氧化碳(CO)、氮氧化物(NO$_\textrm{x}$)、硫氧化物(SO$_\textrm{x}$)等,在這些物質超過(一般的水準)時,便會對我們的身體健康造成不良的影響,甚至引發疾病,大多是呼吸道疾病(),而每個人對空氣汙染的反應不盡相同,這決定於汙染物的多寡、種類、個人的身體素質或是基因等,也因此大多數的空氣汙染與健康的研究多注重在心肺系統與空汙之間的關係,在其他空氣汙染可能會造成的疾病便較難以去界定其關聯性與關係。

除了心肺系統與空氣汙染息息相關外,最容易暴露在空氣中的器官就是皮膚了,無論在室外室內,都是無時無刻接觸著空氣,長期下來不良的空氣對皮膚應具有一定的影響(),在()中有指出懸浮微粒PM2.5、PM10與NO$_\textrm{2}$、SO$_\textrm{2}$與蕁麻疹的就診人數呈現正相關,不過這是對北京地區而言,由於各地區的空氣污染程度不同,其研究結果也侷限在其研究區域,本文將參考該論文()的方法對於高雄地區的空氣汙染與蕁麻疹與過敏的就診人數趨勢進行分析。
\section*{資料來源}
\subsection*{空氣品質監測資料}
自2017年1月1日至2017年12月31日的高雄左營地區空氣品質觀測資料,取自於行政院環境保護署空氣品質監測網(https://taqm.epa.gov.tw/taqm/tw/YearlyDataDownload.aspx),當中包含每日每小時的各項監測濃度,我們取用其中的PM2.5、PM10、NO$_\textrm{2}$、NO、SO$_\textrm{2}$、CO與O$_\textrm{3}$,並計算每日的平均值作為當日監測資料。
\subsection*{蕁麻疹與就診人數資料}
資料來自高雄榮民總醫院(皮膚科),為2017年1月1日至2017年12月31日診斷 ICD-9 代碼為708(蕁麻疹)與995.3(過敏)的每日就診人數資料
\section*{分析方法}
(醫院的就診人數在分布上近似Poisson分布),我們使用廣義加法模型(GAM)來估計空氣汙染對於就診人數之影響,並分為兩階段進行分析,第一階段採用半參數模型,將空氣汙染對於蕁麻疹的影響視為線性相關,亦即空氣汙染每增加一單位;就診人數即增加K單位,而其他因素如時間趨勢的季節效應、當日氣溫、濕度則使用平滑函數來估計在不同數值的影響程度。在第二階段我們也將空氣汙染使用平滑函數來估計在不同的空氣污染程度下對就診人數的影響,亦即在不同的空氣污染程度下對於就診人數並不是線性相關的。而在每個階段分析皆考慮一到七天的滯後效應,事實上(空氣污染的影響並不是即時性的),而考慮一個星期內的滯後效應則是觀察在短期內空氣汙染對於這些疾病的就診情況有無影響,而為了觀察整體的就診人數趨勢與週期,我們也有使用移動平均人數進行分析,即兩天的平均就診人數至七天的平均就診人數

目前皆以單個空汙觀測值作為變數進行分析,未來會考慮多個空汙進行建模。現階段先以各個模型的P-value作為評斷模型的標準來選擇目前的解釋模型。






\section*{univariate gam}
Generalized additive Poisson model
$$
\ln (patient)=Intercept+\beta \times Airpollution+s(temperature)+s(humidity)+s(time)
$$
s= a cyclic cubic regression splines\\
下列依不同的空汙指標分別做單變數 Generalized additive Poisson model,並以時間趨勢、當天的溫度與濕度作為共變量做平滑函數的擬合,下列各空汙列出了不同的滯後天數(row,當天~前七天
)與不同的移動平均天數(colum,當天平均~七天平均)的模型結果(p-value與空汙估計係數)
\clearpage


\subsection*{CO}
下列表格為空氣汙染的滯後效應與移動平均值來做gam 所得到的p-value\\
1-1的值(0.246)為當天的空汙數值與當天就診人數的gam p-value\\
2-1的值(0.683)為前一天的空汙數值與當天就診人數的gam p-value\\
3-1的值(0.002)為前兩天的空汙數值與當天就診人數的gam p-value\\

\begin{table}[h]
\centering
\caption{linear term p-value with lag and moving average data}
\begin{tabular}{rrrrrrrr}
  \hline
 & p.pv & mv2 & mv3 & mv4 & mv5 & mv6 & mv7 \\
  \hline
1 & 0.246 & 0.728 & 0.735 & 0.445 & 0.371 & 0.688 & 0.916 \\
  2 & 0.683 & 0.638 & 0.935 & 0.814 & 0.369 & 0.249 & 0.426 \\
  3 & \textcolor{red}{0.002} & 0.105 & 0.071 & 0.175 & 0.159 & 0.031 & 0.012 \\
  4 & 0.084 & 0.002 & 0.026 & 0.023 & 0.082 & 0.080 & 0.014 \\
  5 & 0.409 & 0.266 & 0.029 & 0.105 & 0.074 & 0.178 & 0.190 \\
  6 & 0.776 & 0.424 & 0.290 & 0.043 & 0.088 & 0.038 & 0.071 \\
  7 & 0.203 & 0.269 & 0.241 & 0.216 & 0.046 & 0.102 & 0.052 \\
  8 & 0.115 & 0.276 & 0.375 & 0.664 & 0.889 & 0.485 & 0.602 \\
   \hline
\end{tabular}
\\row:lag days,col:moving average for the n days
\end{table}

\begin{table}[h]
\centering
\caption{Parametric coefficients with lag and moving average data}
\begin{tabular}{rrrrrrrr}
  \hline
 & beta & mv2 & mv3 & mv4 & mv5 & mv6 & mv7 \\
  \hline
1 & 0.366 & 0.132 & 0.145 & 0.366 & 0.484 & 0.243 & -0.070 \\
  2 & -0.130 & 0.176 & 0.035 & 0.113 & 0.483 & 0.695 & 0.525 \\
  3 & 0.977 & 0.603 & 0.752 & 0.645 & 0.758 & 1.295 & 1.656 \\
  4 & 0.543 & 1.158 & 0.932 & 1.074 & 0.934 & 1.054 & 1.627 \\
  5 & 0.264 & 0.420 & 0.919 & 0.771 & 0.959 & 0.812 & 0.865 \\
  6 & 0.092 & 0.303 & 0.447 & 0.959 & 0.911 & 1.245 & 1.188 \\
  7 & 0.410 & 0.423 & 0.501 & 0.591 & 1.068 & 0.980 & 1.279 \\
  8 & -0.519 & -0.422 & -0.384 & -0.210 & -0.075 & 0.420 & 0.344 \\
   \hline
\end{tabular}
\end{table}
\clearpage
\subsection*{SO2}
\begin{table}[h]
\centering
\caption{linear term p-value with lag and moving average data}
\begin{tabular}{rrrrrrrr}
  \hline
 & p.pv & mv2 & mv3 & mv4 & mv5 & mv6 & mv7 \\
  \hline
1 & 0.017 & 0.016 & 0.219 & 0.009 & 0.016 & 0.095 & 0.129 \\
  2 & 0.042 & 0.495 & 0.195 & 0.446 & 0.022 & 0.027 & 0.105 \\
  3 & 0.000 & 0.100 & 0.002 & 0.001 & 0.009 & 0.000 & 0.000 \\
  4 & 0.054 & 0.131 & 0.887 & 0.124 & 0.071 & 0.163 & 0.010 \\
  5 & 0.316 & 0.029 & 0.681 & 0.488 & 0.534 & 0.332 & 0.529 \\
  6 & 0.338 & 0.171 & 0.030 & 0.980 & 0.369 & 0.756 & 0.477 \\
  7 & 0.394 & 0.127 & 0.070 & 0.012 & 0.439 & 0.106 & 0.613 \\
  8 & 0.379 & 0.957 & 0.516 & 0.246 & 0.071 & 0.758 & 0.382 \\
   \hline
\end{tabular}
\\row:lag days,col:moving average for the n days
\end{table}

\begin{table}[h]
\centering
\caption{Parametric coefficients with lag and moving average data}
\begin{tabular}{rrrrrrrr}
  \hline
 & beta & mv2 & mv3 & mv4 & mv5 & mv6 & mv7 \\
  \hline
1 & 0.054 & 0.083 & 0.053 & 0.126 & 0.126 & 0.095 & 0.092 \\
  2 & -0.059 & 0.024 & 0.056 & 0.037 & 0.121 & 0.126 & 0.098 \\
  3 & 0.081 & 0.057 & 0.130 & 0.153 & 0.138 & 0.224 & 0.236 \\
  4 & -0.056 & 0.053 & 0.006 & 0.074 & 0.095 & 0.080 & 0.155 \\
  5 & -0.028 & -0.085 & 0.018 & -0.034 & 0.033 & 0.055 & 0.038 \\
  6 & -0.026 & -0.052 & -0.098 & -0.001 & -0.048 & 0.018 & 0.043 \\
  7 & -0.023 & -0.058 & -0.081 & -0.125 & -0.041 & -0.093 & -0.031 \\
  8 & 0.021 & -0.002 & -0.029 & -0.057 & -0.097 & -0.018 & -0.053 \\
   \hline
\end{tabular}
\end{table}
\clearpage
\subsection*{O3}
\begin{table}[h]
\centering
\caption{linear term p-value with lag and moving average data}
\begin{tabular}{rrrrrrrr}
  \hline
 & p.pv & mv2 & mv3 & mv4 & mv5 & mv6 & mv7 \\
  \hline
1 & 0.376 & 0.331 & 0.441 & 0.366 & 0.125 & 0.173 & 0.091 \\
  2 & 0.042 & 0.046 & 0.055 & 0.084 & 0.069 & 0.021 & 0.031 \\
  3 & 0.218 & 0.054 & 0.042 & 0.049 & 0.071 & 0.064 & 0.026 \\
  4 & 0.004 & 0.028 & 0.022 & 0.032 & 0.047 & 0.087 & 0.093 \\
  5 & 0.667 & 0.049 & 0.078 & 0.047 & 0.041 & 0.043 & 0.060 \\
  6 & 0.179 & 0.881 & 0.196 & 0.166 & 0.085 & 0.063 & 0.054 \\
  7 & 0.781 & 0.452 & 0.898 & 0.367 & 0.281 & 0.155 & 0.106 \\
  8 & 0.574 & 0.834 & 0.636 & 0.959 & 0.267 & 0.155 & 0.059 \\
   \hline
\end{tabular}
\\row:lag days,col:moving average for the n days
\end{table}

\begin{table}[h]
\centering
\caption{Parametric coefficients with lag and moving average data}
\begin{tabular}{rrrrrrrr}
  \hline
 & beta & mv2 & mv3 & mv4 & mv5 & mv6 & mv7 \\
  \hline
1 & 0.002 & 0.003 & 0.002 & 0.003 & 0.005 & 0.005 & 0.006 \\
  2 & 0.005 & 0.005 & 0.006 & 0.005 & 0.006 & 0.008 & 0.008 \\
  3 & 0.003 & 0.005 & 0.006 & 0.006 & 0.006 & 0.006 & 0.008 \\
  4 & 0.007 & 0.006 & 0.007 & 0.007 & 0.006 & 0.006 & 0.006 \\
  5 & 0.001 & 0.005 & 0.005 & 0.006 & 0.007 & 0.007 & 0.007 \\
  6 & -0.003 & -0.000 & 0.004 & 0.004 & 0.006 & 0.006 & 0.007 \\
  7 & -0.001 & -0.002 & -0.000 & 0.003 & 0.004 & 0.005 & 0.006 \\
  8 & -0.001 & -0.001 & -0.001 & 0.000 & 0.004 & 0.005 & 0.007 \\
   \hline
\end{tabular}
\end{table}
\clearpage
\subsection*{PM2.5}
\begin{table}[h]
\centering
\caption{linear term p-value with lag and moving average data}
\begin{tabular}{rrrrrrrr}
  \hline
 & p.pv & mv2 & mv3 & mv4 & mv5 & mv6 & mv7 \\
  \hline
1 & 0.004 & 0.316 & 0.729 & 0.212 & 0.464 & 0.656 & 0.852 \\
  2 & 0.960 & 0.114 & 0.663 & 0.972 & 0.282 & 0.501 & 0.606 \\
  3 & 0.000 & 0.003 & 0.000 & 0.013 & 0.050 & 0.005 & 0.019 \\
  4 & 0.041 & 0.000 & 0.000 & 0.000 & 0.003 & 0.016 & 0.001 \\
  5 & 0.518 & 0.129 & 0.001 & 0.004 & 0.001 & 0.012 & 0.045 \\
  6 & 0.543 & 0.837 & 0.263 & 0.004 & 0.008 & 0.001 & 0.007 \\
  7 & 0.970 & 0.846 & 0.763 & 0.314 & 0.012 & 0.015 & 0.001 \\
  8 & 0.799 & 0.812 & 0.581 & 0.845 & 0.568 & 0.034 & 0.029 \\
   \hline
\end{tabular}
\\row:lag days,col:moving average for the n days
\end{table}

\begin{table}[h]
\centering
\caption{Parametric coefficients with lag and moving average data}
\begin{tabular}{rrrrrrrr}
  \hline
 & beta & mv2 & mv3 & mv4 & mv5 & mv6 & mv7 \\
  \hline
1 & 0.009 & 0.004 & 0.002 & 0.007 & 0.004 & 0.003 & 0.001 \\
  2 & -0.000 & 0.006 & 0.002 & 0.000 & 0.006 & 0.004 & 0.004 \\
  3 & 0.015 & 0.012 & 0.017 & 0.013 & 0.011 & 0.018 & 0.016 \\
  4 & 0.006 & 0.018 & 0.017 & 0.022 & 0.018 & 0.015 & 0.022 \\
  5 & 0.002 & 0.006 & 0.016 & 0.015 & 0.020 & 0.016 & 0.014 \\
  6 & -0.002 & 0.001 & 0.005 & 0.015 & 0.016 & 0.022 & 0.019 \\
  7 & -0.000 & -0.001 & 0.001 & 0.005 & 0.015 & 0.016 & 0.023 \\
  8 & 0.001 & -0.001 & -0.003 & -0.001 & 0.003 & 0.014 & 0.015 \\
   \hline
\end{tabular}
\end{table}
\clearpage
\subsection*{PM10}
\begin{table}[h]
\centering
\caption{linear term p-value with lag and moving average data}
\begin{tabular}{rrrrrrrr}
  \hline
 & p.pv & mv2 & mv3 & mv4 & mv5 & mv6 & mv7 \\
  \hline
1 & 0.021 & 0.848 & 0.211 & 0.600 & 0.452 & 0.643 & 0.673 \\
  2 & 0.986 & 0.131 & 0.953 & 0.403 & 0.983 & 0.801 & 0.915 \\
  3 & 0.002 & 0.035 & 0.003 & 0.138 & 0.509 & 0.164 & 0.276 \\
  4 & 0.519 & 0.013 & 0.040 & 0.007 & 0.170 & 0.603 & 0.271 \\
  5 & 0.124 & 0.380 & 0.530 & 0.634 & 0.231 & 0.877 & 0.598 \\
  6 & 0.196 & 0.134 & 0.341 & 0.718 & 0.722 & 0.253 & 0.744 \\
  7 & 0.374 & 0.230 & 0.139 & 0.257 & 0.931 & 0.913 & 0.516 \\
  8 & 0.613 & 0.233 & 0.103 & 0.060 & 0.157 & 0.751 & 0.873 \\
   \hline
\end{tabular}
\\row:lag days,col:moving average for the n days
\end{table}

\begin{table}[h]
\centering
\caption{Parametric coefficients with lag and moving average data}
\begin{tabular}{rrrrrrrr}
  \hline
 & beta & mv2 & mv3 & mv4 & mv5 & mv6 & mv7 \\
  \hline
1 & 0.004 & -0.000 & -0.003 & -0.002 & -0.002 & -0.002 & -0.002 \\
  2 & -0.000 & 0.004 & -0.000 & -0.002 & -0.000 & -0.001 & 0.000 \\
  3 & 0.006 & 0.005 & 0.008 & 0.004 & 0.002 & 0.005 & 0.004 \\
  4 & 0.001 & 0.006 & 0.006 & 0.008 & 0.004 & 0.002 & 0.004 \\
  5 & -0.003 & -0.002 & 0.002 & 0.001 & 0.004 & 0.000 & -0.002 \\
  6 & -0.003 & -0.004 & -0.003 & 0.001 & 0.001 & 0.004 & 0.001 \\
  7 & -0.002 & -0.003 & -0.004 & -0.004 & -0.000 & -0.000 & 0.002 \\
  8 & -0.001 & -0.003 & -0.005 & -0.006 & -0.005 & -0.001 & -0.001 \\
   \hline
\end{tabular}
\end{table}
\clearpage
\subsection*{NO}
\begin{table}[h]
\centering
\caption{linear term p-value with lag and moving average data}
\begin{tabular}{rrrrrrrr}
  \hline
 & p.pv & mv2 & mv3 & mv4 & mv5 & mv6 & mv7 \\
  \hline
1 & 0.731 & 0.354 & 0.422 & 0.573 & 0.453 & 0.572 & 0.921 \\
  2 & 0.039 & 0.155 & 0.720 & 0.793 & 0.797 & 0.922 & 0.929 \\
  3 & 0.853 & 0.252 & 0.415 & 0.977 & 0.852 & 0.793 & 0.460 \\
  4 & 0.172 & 0.672 & 0.265 & 0.387 & 0.868 & 0.910 & 0.773 \\
  5 & 0.739 & 0.370 & 0.587 & 0.207 & 0.239 & 0.558 & 0.683 \\
  6 & 0.593 & 0.832 & 0.251 & 0.342 & 0.130 & 0.169 & 0.395 \\
  7 & 0.160 & 0.642 & 0.853 & 0.449 & 0.392 & 0.143 & 0.144 \\
  8 & 0.419 & 0.672 & 0.350 & 0.373 & 0.096 & 0.074 & 0.021 \\
   \hline
\end{tabular}
\\row:lag days,col:moving average for the n days
\end{table}

\begin{table}[h]
\centering
\caption{Parametric coefficients with lag and moving average data}
\begin{tabular}{rrrrrrrr}
  \hline
 & beta & mv2 & mv3 & mv4 & mv5 & mv6 & mv7 \\
  \hline
1 & -0.006 & 0.021 & 0.021 & 0.016 & 0.023 & 0.019 & 0.004 \\
  2 & -0.041 & -0.034 & -0.009 & -0.008 & -0.008 & 0.003 & -0.003 \\
  3 & 0.003 & -0.027 & -0.022 & 0.001 & 0.006 & 0.009 & 0.026 \\
  4 & -0.027 & -0.010 & -0.030 & -0.026 & -0.005 & 0.004 & 0.010 \\
  5 & 0.006 & -0.021 & -0.014 & -0.037 & -0.038 & -0.020 & -0.015 \\
  6 & -0.010 & -0.005 & -0.031 & -0.028 & -0.049 & -0.047 & -0.031 \\
  7 & 0.025 & 0.011 & 0.005 & -0.022 & -0.027 & -0.050 & -0.053 \\
  8 & -0.015 & -0.010 & -0.025 & -0.026 & -0.053 & -0.060 & -0.083 \\
   \hline
\end{tabular}
\end{table}
\clearpage
\subsection*{NO2}
\begin{table}[h]
\centering
\caption{linear term p-value with lag and moving average data}
\begin{tabular}{rrrrrrrr}
  \hline
 & p.pv & mv2 & mv3 & mv4 & mv5 & mv6 & mv7 \\
  \hline
1 & 0.062 & 0.748 & 0.753 & 0.768 & 0.806 & 0.852 & 0.554 \\
  2 & 0.121 & 0.733 & 0.460 & 0.579 & 0.794 & 0.987 & 0.902 \\
  3 & 0.026 & 0.575 & 0.139 & 0.643 & 0.617 & 0.228 & 0.300 \\
  4 & 0.594 & 0.216 & 0.723 & 0.301 & 0.909 & 0.845 & 0.411 \\
  5 & 0.865 & 0.321 & 0.923 & 0.580 & 0.902 & 0.540 & 0.578 \\
  6 & 0.746 & 0.832 & 0.606 & 0.830 & 0.755 & 0.731 & 0.778 \\
  7 & 0.485 & 0.515 & 0.780 & 0.627 & 0.999 & 0.605 & 0.975 \\
  8 & 0.716 & 0.818 & 0.739 & 0.660 & 0.325 & 0.576 & 0.377 \\
   \hline
\end{tabular}
\\row:lag days,col:moving average for the n days
\end{table}

\begin{table}[h]
\centering
\caption{Parametric coefficients with lag and moving average data}
\begin{tabular}{rrrrrrrr}
  \hline
 & beta & mv2 & mv3 & mv4 & mv5 & mv6 & mv7 \\
  \hline
1 & 0.014 & -0.003 & -0.003 & 0.003 & -0.003 & -0.002 & -0.008 \\
  2 & -0.012 & 0.003 & -0.007 & -0.006 & 0.003 & -0.000 & 0.002 \\
  3 & 0.016 & 0.005 & 0.014 & 0.005 & 0.006 & 0.015 & 0.014 \\
  4 & -0.004 & 0.011 & 0.004 & 0.011 & 0.001 & 0.002 & 0.011 \\
  5 & -0.001 & -0.009 & 0.001 & -0.006 & 0.001 & -0.008 & -0.007 \\
  6 & 0.002 & 0.002 & -0.005 & 0.002 & -0.004 & 0.004 & -0.004 \\
  7 & 0.005 & 0.006 & 0.003 & -0.005 & -0.000 & -0.006 & 0.000 \\
  8 & 0.003 & -0.002 & -0.003 & -0.005 & -0.012 & -0.007 & -0.012 \\
   \hline
\end{tabular}
\end{table}
\clearpage



\end{document} 